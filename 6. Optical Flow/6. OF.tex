\documentclass[]{article}
\usepackage{amsmath}
\usepackage{graphicx}
%opening
%\title{}
%\author{}

\begin{document}



\section{Optical Flow}
	\paragraph{Note: These notes are very rough. This section has not been covered to an acceptable level.}
	\subsection{Overview}
		\subsubsection{Motivation}
		\begin{itemize}
			\item Object segmentation (using motion)
			\item Structure recognition
			\item Image alignment
			\item Feature extraction
			\item Video compression
		\end{itemize}
		\subsubsection{Problem definition}
		\begin{itemize}
			\item Compute the pixel movement between frames
			\item Given a pixel at $t$, look for nearby pixes of the same colour in $t+1$
			\item Describe the optical flow, not the motion field (think of a barber's pole)
		\end{itemize}
		\subsubsection{Assumptions}
		\begin{itemize}
			\item Colour constancy: Colour remains constant for a object as it moves
			\item Small motion: In small time increments (frame) only small motion can occur
		\end{itemize}
	\subsection{Horne and Shunck}
		\subsubsection{Solve as optimising an error}
			$$ Error(x,y) = (f_x u + f_y v + f_t)^2 + \lambda (u_x^2 + u_y^2 + v_x^2 + Vy^2) $$
		\textnormal{Seek to find the minimum of this: the first term corresponds to the OF equation and should be zero for the true OF. The second term represents a 'smoothness' constraint i.e. real objects motion should be smooth.}
		\subsubsection{The Aperture Problem}
		\begin{itemize}
			\item If a object is moving, the H and S algorithm can't detect movement of the smooth regions of an object, nor at the edges in the direction parallel to the edge.
		\end{itemize}
	\subsection{Lucas Kanade Algorithm}
		\begin{itemize}
			\item Solves the aperture problem however it fails in areas of large motion
			\item Solve the failure by computing a pyramid (e.g. successive downsampling and iterative compute)
		\end{itemize}
	\subsection{FlowNet: Deep Learning Approach}
		\subsubsection{Overview}
		\begin{itemize}
			\item OF needs precise per-pixel localisation
			\item Requires finding correspondance between 2 pixels
			\item Use a CNN in a sliding window; yet has a problem (compute, per patch nature and ignores global properties) 
		\end{itemize}
		\subsubsection{Architecture}
		\begin{itemize}
			\item Two channels
			\item Two stacked CNN networks: A contracting part takes the input and outputs a small res OF prediction, then an expanding part successively upscales.
			\item There are U-net like skip connections between each halve.
			\item Has an update to handle disparity (think left and right eye) and scene flow
			\item Benchmark on synthetic data
		\end{itemize}
\end{document}
	